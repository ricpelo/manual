%% Generated by Sphinx.
\def\sphinxdocclass{report}
\documentclass[a4paper,12pt,spanish]{sphinxmanual}
\ifdefined\pdfpxdimen
   \let\sphinxpxdimen\pdfpxdimen\else\newdimen\sphinxpxdimen
\fi \sphinxpxdimen=.75bp\relax


\catcode`^^^^00a0\active\protected\def^^^^00a0{\leavevmode\nobreak\ }
\usepackage{cmap}
\usepackage{fontspec}
\usepackage{amsmath,amssymb,amstext}
\usepackage{polyglossia}
\setmainlanguage{spanish}

        \usepackage{libertine}
        \setmonofont[
            Path=../../fonts/,
            BoldFont=FiraMono-Bold.ttf,
            AutoFakeSlant,
            BoldItalicFeatures={FakeSlant},
            Scale=MatchLowercase
        ]{FiraMono-Medium.ttf}
\usepackage[Sonny]{fncychap}
\usepackage[dontkeepoldnames]{sphinx}
\sphinxsetup{
        verbatimwithframe=false,
        noteBorderColor={rgb}{0.208,0.374,0.486},
        noteborder=1pt}
\usepackage{geometry}

% Include hyperref last.
\usepackage{hyperref}
% Fix anchor placement for figures with captions.
\usepackage{hypcap}% it must be loaded after hyperref.
% Set up styles of URL: it should be placed after hyperref.
\urlstyle{same}

\addto\captionsspanish{\renewcommand{\figurename}{Figura}}
\addto\captionsspanish{\renewcommand{\tablename}{Tabla}}
\addto\captionsspanish{\renewcommand{\literalblockname}{Lista}}

\addto\captionsspanish{\renewcommand{\literalblockcontinuedname}{continued from previous page}}
\addto\captionsspanish{\renewcommand{\literalblockcontinuesname}{continues on next page}}

\def\pageautorefname{página}

\setcounter{tocdepth}{1}


        \usepackage[dotinlabels]{titletoc}
        \usepackage{titlesec}
        \titlelabel{\thetitle.\quad}
        \let\sphinxcodeORI\sphinxcode
        \protected\def\sphinxcode #1%
            {{\color{OuterLinkColor}\sphinxcodeORI{#1}}}
        

\title{Manual del proyectista}
\date{noviembre de 2017}
\release{1.0}
\author{Ricardo Pérez López}
\newcommand{\sphinxlogo}{\vbox{}}
\renewcommand{\releasename}{Versión}
\makeindex

\begin{document}

\maketitle
\sphinxtableofcontents
\phantomsection\label{\detokenize{index_latex::doc}}



\chapter{Normativa}
\label{\detokenize{normativa:manual-del-proyectista}}\label{\detokenize{normativa:normativa}}\label{\detokenize{normativa::doc}}\begin{itemize}
\item {} 
Orden 28-09-2011 (BOJA 20-10) de FCT y Proyecto

\item {} 
Orden 29-09-2010 (BOJA 15-10) de evaluación y titulación de Ciclos

\end{itemize}


\chapter{Fases de realización}
\label{\detokenize{fases-de-realizacion:fases-de-realizacion}}\label{\detokenize{fases-de-realizacion::doc}}

\section{Propuesta}
\label{\detokenize{fases-de-realizacion:propuesta}}
Treinta días antes de la fecha prevista para el inicio del proyecto, tienes que presentar una propuesta del proyecto que quieras realizar.

El equipo educativo, en el plazo de una semana, valorará la propuesta y decidirá si lo acepta o no. Para ello, se tiene en cuenta (entre otras cosas), su adecuación a los contenidos trabajados en el Ciclo Formativo y la posibilidad de realización efectiva en los plazos existentes. La decisión te la comunicará el tutor a través de Ágora.

Si tu propuesta no resulta aceptada, tendrás diez días para modificarla, presentar una nueva o aceptar un proyecto propuesto por el Departamento. Ten en cuenta que \sphinxstylestrong{si no haces nada en ese tiempo, se entenderá que renuncias a presentarte, pero te seguirá corriendo convocatoria a no ser que renuncies a tiempo en Jefatura de Estudios}.

Si desde el principio no presentas ninguna propuesta, o la que presentaste resulta rechazada, el Departamento te propondrá un proyecto.

\begin{sphinxadmonition}{danger}{Peligro:}
Recuerda que, si no presentas el proyecto, te correrá la convocatoria a menos que hayas renunciado antes en Jefatura de Estudios.
\end{sphinxadmonition}


\section{Seguimiento}
\label{\detokenize{fases-de-realizacion:seguimiento}}

\section{Presentación}
\label{\detokenize{fases-de-realizacion:presentacion}}

\chapter{Índices y tablas}
\label{\detokenize{index_latex:indices-y-tablas}}


\renewcommand{\indexname}{Índice}
\printindex
\end{document}